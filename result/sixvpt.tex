% !TeX spellcheck = de_DE
\documentclass[22pt]{article}
\title{\Huge {\vspace{-2cm}Six-band}}
\author{\LARGE {Wei Cheng}}
\date{}
\usepackage{color}
\usepackage{verbatim}
\usepackage[UTF8]{ctex}%用于识别中文
\usepackage{underscore}%可以用来正确识别_
\usepackage{graphicx}%用于插入图片
\usepackage{color}
\usepackage{geometry}
\usepackage{braket}
\usepackage{listing}
\geometry{a4paper,scale=0.9}
\usepackage{hyperref}%用于插入链接
\hypersetup{
	colorlinks=true,
	linkcolor=blue,
	filecolor=magenta,      
	urlcolor=blue,
}%插入链接的性质
\usepackage{pythontex}
\usepackage{import}
\usepackage{xifthen}
\usepackage{pdfpages}
\usepackage{transparent}
\usepackage{framed}
\usepackage[dvipsnames,svgnames]{xcolor}
\usepackage[strict]{changepage}
\usepackage{xcolor}
\usepackage{soul}
\usepackage{abstract}
\usepackage{float}
\usepackage{amsmath,amsthm,amssymb,amsfonts, fancyhdr, color, comment, graphicx, environ}
\renewcommand{\abstractname}{\LARGE Abstract}
\usepackage{titlesec}
\definecolor{mygreen}{rgb}{0.8,1,0.73}
\definecolor{linegreencolor}{rgb}{0,0.4,0}
\definecolor{myblue}{rgb}{0.22,0.73,0.91}


\titleformat*{\section}{\LARGE\bfseries}
\titleformat*{\subsection}{\LARGE\bfseries}
\titleformat*{\subsubsection}{\LARGE\bfseries}


\newenvironment{greenbox}{%
	\def\FrameCommand{%
		\hspace{1pt}%
		{\color{linegreencolor}\vrule width 2pt}%
		{\color{mygreen}\vrule width 4pt}%
		\colorbox{mygreen}%
	}%
	\MakeFramed{\advance\hsize-\width\FrameRestore}%
	\noindent\hspace{-4.55pt}% disable indenting first paragraph
	\begin{adjustwidth}{}{7pt}%
		\vspace{2pt}\vspace{2pt}%
	}
	{%
		\vspace{2pt}\end{adjustwidth}\endMakeFramed%
}
\newenvironment{bluebox}{%
	\def\FrameCommand{%
		\hspace{1pt}%
		{\color{myblue}\vrule width 2pt}%
		{\color{myblue}\vrule width 4pt}%
		\colorbox{myblue}%
	}%
	\MakeFramed{\advance\hsize-\width\FrameRestore}%
	\noindent\hspace{-4.55pt}% disable indenting first paragraph
	\begin{adjustwidth}{}{7pt}%
		\vspace{2pt}\vspace{2pt}%
	}
	{%
		\vspace{2pt}\end{adjustwidth}\endMakeFramed%
}


\newcommand{\incfig}[1]{%
	\def\svgwidth{\colum\left( nwidth}
	\import{./images/}{#1.pdf_tex}
}






\begin{document}
	\Large
	\maketitle
	In the basics $\ket{p_z,\uparrow},\ket{p_z,\downarrow},\ket{d_{xz+iyz},\downarrow},\ket{d_{xz-iyz},\uparrow},\ket{d_{xz+iyz},\uparrow},\ket{d_{xz-iyz,\downarrow}}$,考虑$k_z=0$
	\begin{align}
		H=
		\begin{pmatrix}
			M(k) &0&0 &-iA_1k_{-} &-iA_2k_{+}&0\\
			0&M(k)&-iA_1k_{+}&0&0&-iA_2k_{-}\\
			0&iA_1_{_}&-M(k)&0&0&0\\
			iA_1k_{+}&0&0&-M(k)&0&0\\
			iA_2k_{-}&0&0&0&-M(k)+\delta&0\\
			0&iA_2k_{+}&0&0&0&-M(k)+\delta
		\end{pmatrix}
	\end{align}
	做一个基矢变换
	\begin{align}
		U=
		\begin{pmatrix}
			1&0&0&0&0&0\\
			0&0&0&1&0&0\\
			0&0&0&0&1&0\\
			0&1&0&0&0&0\\
			0&0&1&0&0&0\\
			0&0&0&0&0&1\\
		\end{pmatrix}
	\end{align}
	可以得到
	\begin{align}
		H^{'}=
		\begin{pmatrix}
			M(k)&-iA_1k_{-}&-iA_2k_{+}&&&\\
			iA_1k_{+}&-M(k)&0&&&\\
			iA_2k_{-}&0&-M(k)+\delta&&&\\
			&&&M(k)&-iA_1K_{+}&0\\
			&&&iA_1k_{-}&-M(k)&0\\
			&&&iA_2k_{+}&0&-M(k)+\delta
		\end{pmatrix}
	\end{align}
	先考虑其中一个块的内容,其能量本征值满足
	\begin{normalsize}
		\begin{align}
			E^3+(M(k)-\delta)E^2-[M(k)^2+A_1^2k^2+A_2^2k^2]E-M(k)^3+\delta M(k)^2-M(k)[A_1k^2+A_2^2k^2]+\delta A_1^2k^2=0
		\end{align}
	\end{normalsize}
考虑$\delta=0$的情况,可以解得
\begin{align}
	E_1=-M(k) \qquad E_2=-\sqrt{M(k)^2+2A^2k^2}=-E\qquad 
	E_3=\sqrt{M(k)^2+2A^2k^2}
	=E
\end{align}
三个本征态为
\begin{align}
	\ket{\psi_{1}}_e=\frac{1}{\sqrt{2}}
	\begin{pmatrix}
		0\\
		-e^{2i\theta}\\
		1
	\end{pmatrix}
\quad
\ket{\psi_{2}}_e=\frac{1}{a}
\begin{pmatrix}
	i(-M(k)+E)e^{i\theta}\\
	Ake^{2i\theta}\\
	Ak
\end{pmatrix}
\quad
\ket{\psi_{3}}_e=
\frac{1}{b}
\begin{pmatrix}
	-i(M(k)+E)e^{i\theta}\\
	Ake^{2i\theta}\\
	Ak
\end{pmatrix}
\end{align}
在basics$\ket{p_z,\uparrow}_e,\ket{d_{xz-iyz},\uparrow}_e,\ket{d_{xz+iyz},\uparrow}_e,\ket{p_z,\downarrow}_h,\ket{d_{xz+iyz},\downarrow}_h,\ket{d_{xz-iyz},\downarrow}_h$中,可以写出考虑Vortex的BdG哈密顿量为
\begin{align}
	H_{BdG}=
	\begin{pmatrix}
			M(k)-\mu& -iAk_{-}&-iAk_{+} &\Delta&0&0\\
			iAk_{+}&-M(k)-\mu&0&0&\Delta&0\\
			iAk_{-}&0&-M(k)-\mu&0&0&\Delta\\
			\Delta^{\dagger}&0&0&\mu-M(k)&iAk_{-}&iAk_{+}\\
			0&\Delta^{\dagger}&0&-iAk_{+}&\mu+M(k)&0\\
			0&0&\Delta^{\dagger}&-iAk_{-}&0&\mu+M(k)
	\end{pmatrix}
\end{align}
将其变换到H的本征函数空间可以得到
\begin{align}
	\begin{pmatrix}
		E_1-\mu &0&0& \Delta_{11}&\Delta_{12}&\Delta_{13}\\
		0&E_2-\mu&0&\Delta_{21}&\Delta_{22}&0\\
		0&0&E_3-\mu&\Delta_{31}&0&\Delta_{33}\\
		\Delta_{11}^{\dagger}&\Delta_{12}^{\dagger}&\Delta_{13}^{\dagger}&\mu-E_1&0&0\\
		\Delta_{21}^{\dagger}&\Delta_{22}^{\dagger}&0&0&\mu-E_2&0\\
		\Delta_{31}^{\dagger}&0&\Delta_{33}^{\dagger}&0&0&\mu-E_3
	\end{pmatrix}
\end{align}
	其中$\Delta_{ii}=i\Delta_ee^{-i\theta}(\partial_k-\frac{i\partial_{\theta}+A_{\theta}^{11}}{k})\qquad \Delta_{ij}=-i\Delta_ee^{-i\theta}\frac{A_{\theta}^{ij}}{k}$,值得注意的是$\Delta_{23}=\Delta_{32}=0$,即2的电子和3的空穴没有耦合,为了方便起见,将其写在基矢$\ket{\psi_2}_e,\ket{\psi_2}_h,\ket{\psi_1}_e,\ket{\psi_1}_h,\ket{\psi_3}_e,\ket{\psi_3}_h$下,可以得到
	\begin{align}
		\begin{pmatrix}
			E_2-\mu &\Delta_{22} &0&\Delta_{21}&0&0\\
			\Delta_{22}^{\dagger}&\mu-E_2 &\Delta_{21}^{\dagger}&0&0&0\\
			0&\Delta_{12}&E_1-\mu&\Delta_{11}&0&\Delta_{13}\\
			\Delta_{12}^{\dagger}&0&\Delta_{11}^{\dagger}&\mu-E_1&\Delta_{13}^{\dagger}&0\\
			0&0&0&\Delta_{31}&E_3-\mu&\Delta_{33}\\
			0&0&\Delta_{31}^{\dagger}&0&\Delta_{33}^{\dagger}&\mu-E_3
		\end{pmatrix}
	\end{align}
当费米面靠近$E_1,E_2$的时候,可以将其投影到费米面附近的$E_1,E_2$处,由此可以得到
\begin{normalsize}
\begin{align}
	\begin{pmatrix}
		E_2-\mu&i\Delta_ee^{-i\theta}(\partial_k-\frac{i\partial_{\theta}+A_{\theta}^{22}}{k})&0&-i\Delta_ee^{-i\theta}\frac{A_{\theta}^{21}}{k}\\
		i\Delta_ee^{i\theta}(\partial_k+\frac{i\partial_{\theta}+A_{\theta}^{22}}{k})&\mu-E_2&i\Delta_ee^{i\theta}\frac{A_{\theta}^{21}}{k}&0\\
		0&-i\Delta_ee^{-i\theta}\frac{A_{\theta}^{12}}{k}&E_1-\mu&i\Delta_ee^{-i\theta}(\partial_k-\frac{i\partial_{\theta}+A_{\theta}^{11}}{k})\\
		i\Delta_ee^{i\theta}\frac{A_{\theta}^{12}}{k}&0&i\Delta_ee^{i\theta}(\partial_k+\frac{i\partial_{\theta}+A_{\theta}^{11}}{k})&\mu-E_1
	\end{pmatrix}
\end{align}	
\end{normalsize}	
	做一个变换
	\begin{align}
		U=\frac{1}{\sqrt{k}}
		\begin{pmatrix}
			e^{i(j-1)\theta}&&&\\
			&-ie^{ij\theta}&&\\
			&&e^{i(j-1)\theta}&\\
			&&&-ie^{ij\theta}
		\end{pmatrix}
	\end{align}
	可以得到
	\begin{align}
		&\nonumber
		\begin{pmatrix}
			E_2-\mu &\Delta_e(\partial_k+\frac{j-\frac{1}{2}-A_{\theta}^{22}}{k})&0&-\Delta_e\frac{A_{\theta}^{21}}{k}\\
			\Delta_e(\frac{j-\frac{1}{2}-A_{\theta}^{22}}{k}-\partial_k)&\mu-E_2&-\Delta_e\frac{A_{\theta}^{21}}{k}&0\\
			0&-\Delta_e\frac{A_{\theta}^{12}}{k}&E_1-\mu&\Delta_e(\partial_k+\frac{j-\frac{1}{2}-A_{\theta}^{22}}{k})\\
			-\Delta_e\frac{A_{\theta}^{12}}{k}&0&\Delta_e(\frac{j-\frac{1}{2}-A_{\theta}^{22}}{k}-\partial_k)&\mu-E_1
		\end{pmatrix}\\
	&=i\Delta_e\partial_ks_0\tau_y+\frac{1}{2}(E_1+E_2-2\mu)s_0\tau_z+\frac{1}{2}(E_1-E_2)s_z\tau_z+\begin{pmatrix}
		&
	\end{pmatrix}
	\end{align}
前面部分可以看做4*4的Jackiw-Rebbi
\begin{align}
	H_0=i\Delta_e\partial_ks_0\tau_y+\frac{1}{2}(E_1+E_2-2\mu)s_0\tau_z+\frac{1}{2}(E_1-E_2)s_z\tau_z
\end{align}
设其本征态为$\psi(k)$,考虑零能解,可以得到
\begin{align}
	i\Delta_e\partial_ks_0\tau_y\psi(k)=[\frac{1}{2}(E_1+E_2-2\mu)s_0\tau_z+\frac{1}{2}(E_1-E_2)s_z\tau_z]\psi(k)
\end{align}
两边同时乘以$s_0\tau_y$可以得到
\begin{align}
	\partial_k\psi(k)=\frac{1}{\Delta_e}[\frac{1}{2}(E_1+E_2-2\mu)s_0\tau_x+\frac{1}{2}(E_1-E_2)s_z\tau_x]\psi(k)
\end{align}
$\psi(k)$必定是$\frac{1}{2}(E_1+E_2-2\mu)s_0\tau_x+\frac{1}{2}(E_1-E_2)s_z\tau_x$的本征态,可以求得其本征态为
\begin{align}
	\psi_{1}^{+} = \begin{pmatrix}
		1\\
		1\\
		0\\
		0\\
	\end{pmatrix}
	\qquad
	\psi_{1}^{-} = \begin{pmatrix}
		1\\
		-1\\
		0\\
		0\\
	\end{pmatrix}
	\psi_{2}^{+} = \begin{pmatrix}
		0\\
		0\\
		1\\
		1\\
	\end{pmatrix}
	\qquad
	\psi_{2}^{-} = \begin{pmatrix}
		0\\
		0\\
		1\\
		-1\\
	\end{pmatrix}
\end{align}
由此可得
\begin{align}
	\partial_k\psi_1(k)=\eta\frac{E_1-\mu}{\Delta_e}\psi_1(k)\\
	\partial_k\psi_2(k)=\eta\frac{E_2-\mu}{\Delta_e}\psi_2(k)
\end{align}
其中$\eta=\pm$,由此可以得到
\begin{align}
	\psi_1(k)=Ce^{\int^k\eta\frac{E_1-\mu}{\Delta_e}dk^{'}}\\
	\psi_2(k)=Ce^{\int^k\eta\frac{E_2-\mu}{\Delta_e}dk^{'}}
\end{align}
其零能解出现在$E_1-\mu$或$E_2-\mu$改变符号的地方。当这个部分能量为0的时候,系统的能量由
\begin{align}
	\begin{pmatrix}
	  	0&\frac{\Delta_e}{k}(j-\frac{1}{2}-A_{\theta}^{22})&0&-\frac{\Delta_e}{k}A_{\theta}^{21}\\
	  	\frac{\Delta_e}{k}(j-\frac{1}{2}-A_{\theta}^{22})&0&-\frac{\Delta_e}{k}A_{\theta}^{21}&0\\
	  	0&-\frac{\Delta_e}{k}A_{\theta}^{12}&0&\frac{\Delta_e}{k}(j-\frac{1}{2}-A_{\theta}^{11})\\
	  	-\frac{\Delta_e}{k}A_{\theta}^{12}&0&\frac{\Delta_e}{k}(j-\frac{1}{2}-A_{\theta}^{11})&0
	\end{pmatrix}
\end{align}
	将其分块对角,即
	\begin{align}
		&
		\begin{pmatrix}
			\frac{\Delta_e}{k}(j-\frac{1}{2}-A_{\theta}^{22})&-\frac{\Delta_e}{k}A_{\theta}^{21}\\
			-\frac{\Delta_e}{k}A_{\theta}^{12}&\frac{\Delta_e}{k}(j-\frac{1}{2}-A_{\theta}^{11})
		\end{pmatrix}
	&=\frac{\Delta_e}{k}(j-\frac{1}{2}-
	\begin{pmatrix}
		A_{\theta}^{22} &A_{\theta}^{21}\\
		A_{\theta}^{12}&A_{\theta}^{11}
	\end{pmatrix})
	\end{align}
后面这个矩阵可与两条band的SU(2)的Berry phase联系起来,可以证明费米面处的SU(2)Berry phase就是其乘以$2\pi$,从这里可以得出结论就是当费米面处的Berry phase的本征值为$\pi$的奇数倍时,会有零能解。
	可得
	\begin{align}
		A_{\theta}^{22}=-1 \qquad A_{\theta}^{11}=-1 \qquad
		A_{\theta}^{21}=\frac{\sqrt{2}Ak}{\sqrt{(-M(k)+E)^2+2A^2k^2}}
	\end{align}
这个矩阵的本征值为
\begin{align}
	E=-1\pm\frac{\sqrt{2}Ak}{\sqrt{(-M(k)+E)^2+2A^2k^2}}
\end{align}
因为后面那一项明显大于0,小于1,要使其为半整数,只能等于$\frac{1}{2}$,即可以得到
\begin{align}
	\frac{\sqrt{2}Ak}{\sqrt{(-M(k)+E)^2+2A^2k^2}}=\frac{1}{2}
\end{align}
可以得到
\begin{align}
	3M^2(k)-2A^2k^2=0
\end{align}
可以取参数与数值计算的对应,但目前对照得还是有点问题,可能是某个地方有问题,即这两条band相变的能量不一样,一个是$-M(k)$,另外一个是$-2M(k)$。
\begin{align}
	\ket{\psi_{1}}_e=\frac{1}{\sqrt{2}}
	\begin{pmatrix}
		0\\
		-1\\
		1
	\end{pmatrix}
	\quad
	\ket{\psi_{2}}_e=\frac{1}{a}
	\begin{pmatrix}
		i(-M(k)+E)e^{i\theta}\\
		Ak\\
		Ak
	\end{pmatrix}
	\quad
	\ket{\psi_{3}}_e=
	\frac{1}{b}
	\begin{pmatrix}
		-i(M(k)+E)e^{i\theta}\\
		Ak\\
		Ak
	\end{pmatrix}
\end{align}
做同样的变换之后容易发现$\Delta_{21}=\Delta_{31}=0$即耦合项全部没了,因此变换到本征基矢表象下后可以完全的分块对角。而且此时容易发现$A_{\theta}^{11}=0$不可能为半整数,因此1这条带不会有拓扑相变。此时可以计算
\begin{align}
	A_{\theta}^{22}=-\frac{(-M(k)+E)^2}{(-M(k)+E)^2+2A^2k^2}
\end{align}
要想使其为半整数,容易分析知必为$-\frac{1}{2}$由此可以得到$M(k)=0$带入参数$M0=-1,M1=1,A=0.5$可以得到相变点的$k=\pm 1$,相变的能量为$-\sqrt{2}0.5= 0.707$,对于$E_3$同样分析也可以得到其相变能量为0.707,这一点与数值计算吻合。

\section{Conclusion}
在只有两个相变点的时候,与数值计算的结果能够对应起来,有四个相变点的时候对应有问题,数值上当$\Delta_{so}=0$的时候,四个相变点是两两重合的,但是目前我算的有点错位,即中间那条带的相变能量与另外两条不一样,我感觉上应该是计算上的问题。

\end{document}