\documentclass[22pt]{article}
\title{\Huge {\vspace{-2cm}Six-band}}
\author{\LARGE {Wei Cheng}}
\date{}
\usepackage{color}
\usepackage{verbatim}
\usepackage[UTF8]{ctex}%用于识别中文
\usepackage{underscore}%可以用来正确识别_
\usepackage{graphicx}%用于插入图片
\usepackage{color}
\usepackage{geometry}
\usepackage{braket}
\usepackage{listing}
\geometry{a4paper,scale=0.9}
\usepackage{hyperref}%用于插入链接
\hypersetup{
	colorlinks=true,
	linkcolor=blue,
	filecolor=magenta,      
	urlcolor=blue,
}%插入链接的性质
\usepackage{pythontex}
\usepackage{import}
\usepackage{xifthen}
\usepackage{pdfpages}
\usepackage{transparent}
\usepackage{framed}
\usepackage[dvipsnames,svgnames]{xcolor}
\usepackage[strict]{changepage}
\usepackage{xcolor}
\usepackage{soul}
\usepackage{abstract}
\usepackage{float}
\usepackage{amsmath,amsthm,amssymb,amsfonts, fancyhdr, color, comment, graphicx, environ}
\renewcommand{\abstractname}{\LARGE Abstract}
\usepackage{titlesec}
\definecolor{mygreen}{rgb}{0.8,1,0.73}
\definecolor{linegreencolor}{rgb}{0,0.4,0}
\definecolor{myblue}{rgb}{0.22,0.73,0.91}


\titleformat*{\section}{\LARGE\bfseries}
\titleformat*{\subsection}{\LARGE\bfseries}
\titleformat*{\subsubsection}{\LARGE\bfseries}


\newenvironment{greenbox}{%
	\def\FrameCommand{%
		\hspace{1pt}%
		{\color{linegreencolor}\vrule width 2pt}%
		{\color{mygreen}\vrule width 4pt}%
		\colorbox{mygreen}%
	}%
	\MakeFramed{\advance\hsize-\width\FrameRestore}%
	\noindent\hspace{-4.55pt}% disable indenting first paragraph
	\begin{adjustwidth}{}{7pt}%
		\vspace{2pt}\vspace{2pt}%
	}
	{%
		\vspace{2pt}\end{adjustwidth}\endMakeFramed%
}
\newenvironment{bluebox}{%
	\def\FrameCommand{%
		\hspace{1pt}%
		{\color{myblue}\vrule width 2pt}%
		{\color{myblue}\vrule width 4pt}%
		\colorbox{myblue}%
	}%
	\MakeFramed{\advance\hsize-\width\FrameRestore}%
	\noindent\hspace{-4.55pt}% disable indenting first paragraph
	\begin{adjustwidth}{}{7pt}%
		\vspace{2pt}\vspace{2pt}%
	}
	{%
		\vspace{2pt}\end{adjustwidth}\endMakeFramed%
}


\newcommand{\incfig}[1]{%
	\def\svgwidth{\colum\left( nwidth}
	\import{./images/}{#1.pdf_tex}
}






\begin{document}
	\Large
	\maketitle
	In the basics $\ket{p_z,\uparrow},\ket{p_z,\downarrow},\ket{d_{xz+iyz},\downarrow},\ket{d_{xz-iyz},\uparrow},\ket{d_{xz+iyz},\uparrow},\ket{d_{xz-iyz,\downarrow}}$,考虑$k_z=0$
	\begin{align}
		H=
		\begin{pmatrix}
			M(k) &0&0 &-iA_1k_{-} &-iA_2k_{+}&0\\
			0&M(k)&-iA_1k_{+}&0&0&-iA_2k_{-}\\
			0&iA_1_{_}&-M(k)&0&0&0\\
			iA_1k_{+}&0&0&-M(k)&0&0\\
			iA_2k_{-}&0&0&0&-M(k)+\delta&0\\
			0&iA_2k_{+}&0&0&0&-M(k)+\delta
		\end{pmatrix}
	\end{align}
	做一个基矢变换
	\begin{align}
		U=
			\begin{pmatrix}
			1&0&0&0&0&0\\
			0&0&0&1&0&0\\
			0&0&0&0&1&0\\
			0&1&0&0&0&0\\
			0&0&1&0&0&0\\
			0&0&0&0&0&1\\
			\end{pmatrix}
	\end{align}
	可以得到
	\begin{align}
		H^{'}=
		\begin{pmatrix}
			M(k)&-iA_1k_{-}&-iA_2k_{+}&&&\\
			iA_1k_{+}&-M(k)&0&&&\\
			iA_2k_{-}&0&-M(k)+\delta&&&\\
			&&&M(k)&-iA_1K_{+}&0\\
			&&&iA_1k_{-}&-M(k)&0\\
			&&&iA_2k_{+}&0&-M(k)+\delta
		\end{pmatrix}
	\end{align}
先考虑其中一个块的内容,其能量本征值满足
\begin{normalsize}
\begin{align}
	E^3+(M(k)-\delta)E^2-[M(k)^2+A_1^2k^2+A_2^2k^2]E-M(k)^3+\delta M(k)^2-M(k)[A_1k^2+A_2^2k^2]+\delta A_1^2k^2=0
\end{align}
\end{normalsize}
考虑vortex,并写到动量空间的极坐标下之后可以得到
\begin{align}
	H_{BdG}=
	\begin{pmatrix}
		H_k-\mu & i\Delta_ee^{-i\theta}(\partial_k-\frac{i}{k}\partial_{\theta})\\
		i\Delta_ee^{i\theta}(\partial_k+\frac{i}{k}\partial_{\theta})&\mu-H_k
	\end{pmatrix}
\end{align}
将其投影到$H_k$的本征空间
可以得到

\begin{align}
	\begin{pmatrix}
	E_1-\mu&&&\Delta_{11}&C_{12}&C_{13}\\
	&E_2-\mu&&C_{21}&\Delta_{22}&C_{23}\\
	&&E_3-\mu&C_{31}&C_{32}&\Delta_{33}\\
	\Delta_{11}^{\dagger}&C_{21}^{\dagger}&C_{31}^{\dagger}&\mu-E_1&&\\
	C_{12}^{\dagger}&\Delta_{22}^{\dagger}&C_{32}^{\dagger}&&\mu-E_2&\\
	C_{13}^{\dagger}&C_{23}^{\dagger}&\Delta_{33}^{\dagger}&&&\mu-E_3\\
	\end{pmatrix}
\end{align}
其中$\Delta_{ii}=i\Delta_ee^{-i\theta}(\partial_k-iA_k^{11}-\frac{i\partial_{\theta}+A_{\theta}^{11}}{k})$,$C_{ij}=\Delta_ee^{-i\theta}(A_k^{ij}-\frac{i}{k}A_{\theta}^{ij})$
将其变换到basics$\ket{\psi_{1}_e},\ket{\psi_{1}_h},\ket{\psi_{2}_e},\ket{\psi_{2}_h},\ket{\psi_{2}_e},\ket{\psi_{2}_h}$
可以得到
\begin{small}
\begin{align}
	\begin{pmatrix}
		E_1-\mu &\Delta_{11} &0&C_{12}&0&C_{13}\\
		\Delta_{11}^{\dagger}&\mu-E_1&C_{21}^{\dagger}&0&C_{31}^{\dagger}&0\\
		0&C_{21}&E_2-\mu&\Delta_{22}&0&C_{23}\\
		C_{12}^{\dagger}&0&\Delta_{22}^{\dagger}&\mu-E_2&C_{32}^{\dagger}&0\\
		0&C_{31}&0&C_{32}&E_3-\mu&\Delta_{33}\\
		C_{13}^{\dagger}&0&C_{23}^{\dagger}&0&\Delta_{33}^{\dagger}&\mu-E_3
	\end{pmatrix}
\end{align}
\end{small}
根据中岛变换,将其变换成上面$4\cdot 4$和一个$2\cdot 2$的块对角之中,首先将H分解为$H=H_0+H^1+H^2$,其中$H_0$表示对角部分,$H^1$表示非对角但是在对角块的部分,$H^2$表示非对角块部分。由$H^{'}=e^SHe^{-S}$,考虑一阶近似,可以得到变换矩阵为
\begin{align}
	S_{ml}^{(1)}=\frac{-H^2_{ml}}{E_m-E_l}
\end{align}

同样去一阶近似可以得到
\begin{align}
	H^{'}=H_0+H^1+[H^2,S^1]
\end{align}
首先考虑上面$4*4$的部分,
\begin{align}
	H_{04}=
	\begin{pmatrix}
		E_1-\mu & \Delta_{11} &0 &C_{12}\\
		\Delta_{11}^{\dagger}&\mu-E_1&C_{21}^{\dagger}&0\\
		0&C_{21}&E_2-\mu&\Delta_{22}\\
		C_{12}^{\dagger}&0&\Delta_{22}^{\dagger}&\mu-E_2
	\end{pmatrix}
\end{align}
一阶微扰项为
\begin{align}
	H_{04}^{'}=
	\begin{pmatrix}
		&&\frac{C_{13}C_{23}^{\dagger}(E_2-E_1)}{(E_1+E_3)(E_2+E_3)}&\\
		&&&\frac{C_{31}^{\dagger}C_{32}(E_2-E_1)}{(E_1+E_3)(E_2+E_3)}&&&\\
		\frac{C_{13}^{\dagger}C_{23}(E_2-E_1)}{(E_1+E_3)(E_2+E_3)}&&&\\
		&\frac{C_{32}^{\dagger}C_{31}(E_2-E_1)}{(E_1+E_3)(E_2+E_3)}
	\end{pmatrix}
\end{align}
首先求解$H_{04}$的本征态
\begin{small}
\begin{align}
	\begin{pmatrix}
	E_1-\mu & i\Delta_ee^{-i\theta}(\partial_k-iA_k^{11}-\frac{i\partial_{\theta}+A_{\theta}^{11}}{k})&0&\Delta_ee^{-i\theta}(A_k^{12}-\frac{i}{k}A_{\theta}^{12})\\
	i\Delta_ee^{i\theta}(\partial_k-iA_k^{11}+\frac{i\partial_{\theta}+A_{\theta}^{11}}{k})&\mu-E_1&\Delta_ee^{i\theta}(A_k^{12}+\frac{i}{k}A_{\theta}^{12})&0\\
	0&\Delta_ee^{-i\theta}(A_k^{21}-\frac{i}{k}A_{\theta}^{21})&E_2-\mu&i\Delta_ee^{-i\theta}(\partial_k-iA_k^{22}-\frac{i\partial_{\theta}+A_{\theta}^{22}}{k})\\
	\Delta_ee^{i\theta}(A_k^{21}+\frac{i}{k}A_{\theta}^{21})&0&i\Delta_ee^{i\theta}(\partial_k-iA_k^{22}+\frac{i\partial_{\theta}+A_{\theta}^{22}}{k})&\mu-E_2
	\end{pmatrix}
\end{align}
\end{small}
做一个变换
\begin{align}
	U=\frac{1}{\sqrt{k}}
	\begin{pmatrix}
		e^{i(j-1)\theta}&&&\\
		&-ie^{ij\theta}&&\\
		&&e^{i(j-1)\theta}&\\
		&&&-ie^{ij\theta}
	\end{pmatrix}
\end{align}
变换后可以得到
\begin{small}
\begin{align}
	U^{\dagger}HU&=
	\begin{pmatrix}
		E_1-\mu &\Delta_e(\partial_k-iA_k^{11}+\frac{j-\frac{1}{2}-A_{\theta}^{11}}{k})&0&-i\Delta_e(A_k^{12}-\frac{i}{k}A_{\theta}^{12})\\
		\Delta_e(\frac{j-\frac{1}{2}-A_{\theta}^{11}}{k}+iA_k^{11}-\partial_k)&\mu-E_1&i\Delta_e(A_k^{12}+\frac{i}{k}A_{\theta}^{12})&0\\
		0&-i\Delta_e(A_k^{21}-\frac{i}{k}A_{\theta}^{21})&E_2-\mu&\Delta_e(\partial_k-iA_k^{22}+\frac{j-\frac{1}{2}-A_{\theta}^{22}}{k})\\
		i\Delta_e(A_k^{21}+\frac{i}{k}A_{\theta}^{21})&0&\Delta_e(\frac{j-\frac{1}{2}-A_{\theta}^{22}}{k}+iA_{k}^{22}-\partial_k)&\mu-E_2
	\end{pmatrix}\\
&=i\Delta_e\partial_ks_0\tau_y+\frac{1}{2}(E_1+E_2-2\mu)s_0\tau_z+\frac{1}{2}(E_1-E_2)s_z\tau_z+\begin{pmatrix}
	&
\end{pmatrix}
\end{align}
\end{small}
其中前两项似乎可以看成四维的JRmodel
\begin{align}
	H_0=i\Delta_e\partial_ks_0\tau_y+\frac{1}{2}(E_1+E_2-2\mu)s_0\tau_z+\frac{1}{2}(E_1-E_2)s_z\tau_z
\end{align}
设其本征态为$\psi(k)$,考虑零能解,可以得到
\begin{align}
	i\Delta_e\partial_ks_0\tau_y\psi(k)=[\frac{1}{2}(E_1+E_2-2\mu)s_0\tau_z+\frac{1}{2}(E_1-E_2)s_z\tau_z]\psi(k)
\end{align}
两边同时乘以$s_0\tau_y$可以得到
\begin{align}
	\partial_k\psi(k)=\frac{1}{\Delta_e}[\frac{1}{2}(E_1+E_2-2\mu)s_0\tau_x+\frac{1}{2}(E_1-E_2)s_z\tau_x]\psi(k)
\end{align}
$\psi(k)$必定是$\frac{1}{2}(E_1+E_2-2\mu)s_0\tau_x+\frac{1}{2}(E_1-E_2)s_z\tau_x$的本征态,可以求得其本征态为
\begin{align}
	\psi_{1}^{+} = \begin{pmatrix}
		1\\
		1\\
		0\\
		0\\
	\end{pmatrix}
\qquad
\psi_{1}^{-} = \begin{pmatrix}
	1\\
	-1\\
	0\\
	0\\
\end{pmatrix}
	\psi_{2}^{+} = \begin{pmatrix}
	0\\
	0\\
	1\\
	1\\
\end{pmatrix}
\qquad
\psi_{2}^{-} = \begin{pmatrix}
	0\\
	0\\
	1\\
	-1\\
\end{pmatrix}
\end{align}
由此可得
\begin{align}
	\partial_k\psi_1(k)=\eta\frac{E_1-\mu}{\Delta_e}\psi_1(k)\\
	\partial_k\psi_2(k)=\eta\frac{E_2-\mu}{\Delta_e}\psi_2(k)
\end{align}
其中$\eta=\pm$,由此可以得到
\begin{align}
		\psi_1(k)=Ce^{\int^k\eta\frac{E_1-\mu}{\Delta_e}dk^{'}}\\
		\psi_2(k)=Ce^{\int^k\eta\frac{E_2-\mu}{\Delta_e}dk^{'}}
\end{align}
其零能解出现在$E_1-\mu$或$E_2-\mu$改变符号的地方。此时再来求解这个矩阵的本征值
\begin{align}
	\begin{pmatrix}
		0&\Delta_e(\frac{j-\frac{1}{2}-A_{\theta}^{11}}{k}-iA_k^{11})&0&-i\Delta_e(A_k^{12}-\frac{i}{k}A_{\theta}^{12})\\
		\Delta_e(\frac{j-\frac{1}{2}-A_{\theta}^{11}}{k}+iA_k^{11})&0&i\Delta_e(A_k^{12}+\frac{i}{k}A_{\theta}^{12})&0\\
		0&-i\Delta_e(A_k^{21}-\frac{i}{k}A_{\theta}^{21})&0&\Delta_e(\frac{j-\frac{1}{2}-A_{\theta}^{22}}{k}-iA_k^{22})\\
		i\Delta_e(A_k^{21}+\frac{i}{k}A_{\theta}^{21})&0&\Delta_e(\frac{j-\frac{1}{2}-A_{\theta}^{22}}{k}+iA_k^{22})
	\end{pmatrix}
\end{align}
将其分块对角,即求解
\begin{align}
	&
	\begin{pmatrix}
	\Delta_e(\frac{j-\frac{1}{2}-A_{\theta}^{11}}{k}-iA_k^{11})&-i\Delta_e(A_k^{12}-\frac{i}{k}A_{\theta}^{12})\\
	-i\Delta_e(A_k^{21}-\frac{i}{k}A_{\theta}^{21})&
	\Delta_e(\frac{j-\frac{1}{2}-A_{\theta}^{22}}{k}-iA_k^{22})
	\end{pmatrix}\\
&=\frac{\Delta_e}{k}(j-\frac{1}{2})\sigma_0+\Delta_e
\begin{pmatrix}
	-\frac{A_{\theta}^{11}}{k}-iA_k^{11}&-\frac{A_{\theta}^{12}}{k}-iA_k^{12}\\
	-\frac{A_{\theta}^{21}}{k}-iA_k^{21}&-\frac{A_{\theta}^{22}}{k}-iA_k^{22}
\end{pmatrix}
\end{align}
各向同性的时候费米面处SU(2)的Berry phase为
\begin{align}
	\phi_{FS}&=\oint_{FS}\vec{A_{ij}}d\vec{k}\\
	&=\int_0^{2\pi}kd\theta \bra{\psi_i}\frac{\partial_{\theta}}{k}\ket{\psi_j}
	\\
	&=2\pi A_{\theta}^{ij}
\end{align}
其中$A_k$无法通过规范变换全部消除,即除了Berry phase似乎还有其他项,但这些项是否为0我还无法确定。
\section{Conclusion}
在$k_z=0$的时候,将电子部分哈密顿量化成了3*3分块对角,然后考虑超导以及vortex,将其写到particle-hole space,并且投影到电子哈密顿量的本征态上,然后将6*6的哈密顿量用中岛变换将非对角块的部分变换到块对角部分,目前只考虑到一阶近似。先考虑4*4这一块部分,首先不考虑中岛变换过来的微扰项,这一部分似乎可以化成4*4的Jackiw-Rebbi,然后按照能量可以变成两套2*2的Jackiw-Rebbi。然后进一步考虑其他项的时候,发现除了SU(2)的Berry phase还有一些$A_k$项,感觉无法消掉。

\end{document}